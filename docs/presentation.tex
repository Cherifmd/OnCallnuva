% =============================================================================
% OnCallNova — Plateforme de Gestion d'Incidents (Type PagerDuty) 100% Locale
% Hackathon OpCell — Document d'Architecture & Justifications Techniques
% =============================================================================
\documentclass[a4paper,11pt]{article}

% ---------- Packages ----------
\usepackage[utf8]{inputenc}
\usepackage[T1]{fontenc}
\usepackage[french]{babel}
\usepackage{geometry}
\geometry{margin=2cm}
\usepackage{graphicx}
\usepackage{xcolor}
\usepackage{tikz}
\usetikzlibrary{
  shapes.geometric, arrows.meta, positioning, fit,
  backgrounds, calc, decorations.markings, shadows,
  matrix, patterns, chains, scopes
}
\usepackage{tabularx}
\usepackage{booktabs}
\usepackage{enumitem}
\usepackage{listings}
\usepackage{fancyhdr}
\usepackage{hyperref}
\usepackage{fontawesome5}
\usepackage{tcolorbox}
\tcbuselibrary{skins,breakable}
\usepackage{multicol}
\usepackage{caption}
\usepackage{float}
\usepackage{amssymb}
\usepackage{pifont}

% ---------- Couleurs ----------
\definecolor{primary}{HTML}{1E88E5}
\definecolor{secondary}{HTML}{43A047}
\definecolor{accent}{HTML}{FB8C00}
\definecolor{danger}{HTML}{E53935}
\definecolor{dark}{HTML}{212121}
\definecolor{light}{HTML}{F5F5F5}
\definecolor{grpcblue}{HTML}{244C5A}
\definecolor{redisbg}{HTML}{D32F2F}
\definecolor{postgresbg}{HTML}{336791}
\definecolor{traefikbg}{HTML}{00BCD4}
\definecolor{prometheusbg}{HTML}{E6522C}
\definecolor{grafanabg}{HTML}{F46800}
\definecolor{jaegerbg}{HTML}{66CFE3}
\definecolor{lokibg}{HTML}{F9A825}
\definecolor{codebg}{HTML}{282C34}

% ---------- Style listings ----------
\lstset{
  basicstyle=\ttfamily\small,
  backgroundcolor=\color{codebg},
  commentstyle=\color{secondary},
  keywordstyle=\color{primary}\bfseries,
  stringstyle=\color{accent},
  breaklines=true,
  frame=single,
  rulecolor=\color{dark!30},
  numbers=left,
  numberstyle=\tiny\color{dark!40},
  xleftmargin=1.5em
}

% ---------- Style tcolorbox ----------
\newtcolorbox{techbox}[1]{
  colback=primary!5, colframe=primary!80,
  title={\faIcon{microchip}~#1}, fonttitle=\bfseries,
  breakable, enhanced, shadow={2mm}{-1mm}{0mm}{black!20}
}
\newtcolorbox{bonusbox}[1]{
  colback=accent!5, colframe=accent!80,
  title={\faIcon{star}~#1}, fonttitle=\bfseries,
  breakable, enhanced, shadow={2mm}{-1mm}{0mm}{black!20}
}
\newtcolorbox{justifbox}[1]{
  colback=secondary!5, colframe=secondary!80,
  title={\faIcon{check-circle}~#1}, fonttitle=\bfseries,
  breakable, enhanced
}
\newtcolorbox{warnbox}[1]{
  colback=danger!5, colframe=danger!80,
  title={\faIcon{shield-alt}~#1}, fonttitle=\bfseries,
  breakable, enhanced
}

% ---------- En-tête / Pied de page ----------
\pagestyle{fancy}
\fancyhf{}
\fancyhead[L]{\textcolor{primary}{\textbf{OnCallNova}}}
\fancyhead[R]{\textcolor{dark!60}{Hackathon OpCell}}
\fancyfoot[C]{\thepage}
\renewcommand{\headrulewidth}{1pt}
\renewcommand{\headrule}{\hbox to\headwidth{\color{primary}\leaders\hrule height \headrulewidth\hfill}}

% ---------- Commandes utiles ----------
\newcommand{\cmark}{\textcolor{secondary}{\ding{51}}}
\newcommand{\xmark}{\textcolor{danger}{\ding{55}}}
\newcommand{\tech}[1]{\texttt{\textcolor{primary}{#1}}}

% =============================================================================
\begin{document}

% ===================== PAGE DE TITRE =====================
\begin{titlepage}
\begin{center}
\vspace*{2cm}

\begin{tikzpicture}
  \node[circle, fill=primary, minimum size=3cm, text=white, font=\Huge\bfseries] {ON};
  \node[below=0.1cm, font=\Huge\bfseries\color{dark}] at (0,-1.7) {OnCallNova};
\end{tikzpicture}

\vspace{1cm}
{\LARGE\color{dark}\textbf{Plateforme de Gestion d'Incidents}}\\[0.3cm]
{\Large\color{primary} Type PagerDuty — 100\% Locale}\\[1.5cm]

\begin{tikzpicture}
  \foreach \i/\icon/\lbl in {
    0/server/5 Microservices,
    1/network-wired/gRPC \& REST,
    2/database/PostgreSQL + Redis,
    3/shield-alt/JWT Auth,
    4/chart-line/Observabilité
  }{
    \node[draw=primary!60, rounded corners=8pt, fill=primary!8,
          minimum width=2.8cm, minimum height=1.2cm, align=center,
          font=\small] at (\i*3.2,0)
      {\faIcon{\icon}\\[2pt]\lbl};
  }
\end{tikzpicture}

\vspace{2cm}
{\large\color{dark!70} Hackathon OpCell — Document d'Architecture}\\[0.5cm]
{\color{dark!50} \today}

\vfill
\end{center}
\end{titlepage}

% ===================== TABLE DES MATIÈRES =====================
\tableofcontents
\newpage

% =============================================================================
% SECTION 1 : VUE D'ENSEMBLE
% =============================================================================
\section{Vue d'Ensemble du Projet}

\begin{techbox}{Résumé Exécutif}
\textbf{OnCallNova} est une plateforme de gestion d'incidents de production déployable
\textbf{100\% en local} via Docker Compose. Elle reproduit les fonctionnalités
essentielles d'un outil comme PagerDuty~: ingestion d'alertes, corrélation
intelligente, gestion du cycle de vie des incidents, planification d'astreintes
et tableau de bord temps réel — le tout orchestré par \textbf{5 microservices}
communiquant en \textbf{gRPC} et exposés derrière un API Gateway \textbf{Traefik}.
\end{techbox}

\subsection{Objectifs Fonctionnels}

\begin{enumerate}[label=\textbf{\arabic*.}, leftmargin=2em]
  \item \textbf{Ingestion \& Corrélation} — Recevoir des alertes multi-sources et
        les regrouper en incidents via un algorithme de fenêtre temporelle.
  \item \textbf{Gestion du Cycle de Vie} — Créer, acquitter, résoudre, escalader
        les incidents avec traçabilité complète.
  \item \textbf{Planification d'Astreintes} — Définir des équipes, des rotations et
        identifier l'ingénieur de garde en temps réel.
  \item \textbf{Observabilité} — Métriques Prometheus, dashboards Grafana, tracing
        distribué Jaeger.
  \item \textbf{Sécurité} — Authentification JWT, rate limiting Traefik, chiffrement
        des secrets.
\end{enumerate}

% =============================================================================
% SECTION 2 : ARCHITECTURE GLOBALE
% =============================================================================
\section{Architecture Globale}

\subsection{Diagramme de Topologie des Services}

\begin{figure}[H]
\centering
\begin{tikzpicture}[
  node distance=1.2cm and 2.5cm,
  every node/.style={font=\small},
  service/.style={
    rectangle, rounded corners=6pt, draw=#1!80, fill=#1!12,
    minimum width=3.2cm, minimum height=1.4cm, align=center,
    font=\small\bfseries, drop shadow={shadow xshift=1pt, shadow yshift=-1pt}
  },
  infra/.style={
    rectangle, rounded corners=4pt, draw=#1!80, fill=#1!15,
    minimum width=2.8cm, minimum height=1cm, align=center,
    font=\footnotesize\bfseries
  },
  arrow/.style={-{Stealth[length=6pt]}, thick, #1},
  biarrow/.style={{Stealth[length=6pt]}-{Stealth[length=6pt]}, thick, #1},
  label/.style={font=\scriptsize\color{dark!60}, midway, above, sloped}
]

% --- Couche Client ---
\node[service=dark] (client) {\faIcon{desktop}\\Client / Navigateur};

% --- API Gateway ---
\node[service=traefikbg, below=1.5cm of client] (traefik)
  {\faIcon{project-diagram}\\Traefik v3\\{\scriptsize Port 80 | Rate Limit}};

% --- Microservices ---
\node[service=primary, below left=2cm and 3cm of traefik] (alert)
  {\faIcon{bell}\\Alert Ingestion\\{\scriptsize :8001 | :50051}};
\node[service=danger, below=2cm of traefik] (incident)
  {\faIcon{fire}\\Incident Mgmt\\{\scriptsize :8002 | :50052}};
\node[service=secondary, below right=2cm and 3cm of traefik] (oncall)
  {\faIcon{user-clock}\\OnCall Service\\{\scriptsize :8003 | :50053}};

\node[service=accent, left=1.5cm of alert] (webui)
  {\faIcon{window-maximize}\\Web UI\\{\scriptsize :8080}};
\node[service=prometheusbg, right=1.5cm of oncall] (metrics)
  {\faIcon{tachometer-alt}\\Metrics Exporter\\{\scriptsize :9090}};

% --- Flèches Client → Traefik → Services ---
\draw[arrow=dark] (client) -- (traefik);
\draw[arrow=traefikbg] (traefik) -- node[label] {REST} (alert);
\draw[arrow=traefikbg] (traefik) -- node[label] {REST} (incident);
\draw[arrow=traefikbg] (traefik) -- node[label] {REST} (oncall);
\draw[arrow=traefikbg] (traefik) -| node[label, near start] {REST} (webui);
\draw[arrow=traefikbg] (traefik) -| node[label, near start] {REST} (metrics);

% --- Inter-service gRPC ---
\draw[biarrow=grpcblue, dashed] (alert) -- node[label] {gRPC} (incident);
\draw[biarrow=grpcblue, dashed] (incident) -- node[label] {gRPC} (oncall);

% --- Couche Data ---
\node[infra=postgresbg, below=3.5cm of alert] (pg)
  {\faIcon{database}~PostgreSQL 16};
\node[infra=redisbg, below=3.5cm of oncall] (redis)
  {\faIcon{bolt}~Redis 7};

\draw[arrow=postgresbg] (alert) -- (pg);
\draw[arrow=postgresbg] (incident) -- (pg);
\draw[arrow=postgresbg] (oncall) -- (pg);
\draw[arrow=redisbg] (alert) -- (redis);
\draw[arrow=redisbg] (incident) -- (redis);
\draw[arrow=redisbg] (oncall) -- (redis);

% --- Couche Observabilité ---
\node[infra=prometheusbg, below=3.5cm of incident] (prom)
  {\faIcon{chart-bar}~Prometheus};
\node[infra=grafanabg, right=0.8cm of prom] (grafana)
  {\faIcon{chart-area}~Grafana};
\node[infra=jaegerbg, left=0.8cm of prom] (jaeger)
  {\faIcon{search}~Jaeger};

\draw[arrow=prometheusbg, dotted] (prom) -- (grafana);
\draw[arrow=prometheusbg, dotted] (metrics) |- (prom);
\draw[arrow=jaegerbg, dotted] (alert.south) -- ++(0,-0.8) -| (jaeger);
\draw[arrow=jaegerbg, dotted] (incident.south) -- ++(0,-0.5) -| (jaeger);

% --- Légende ---
\node[below=6.5cm of traefik, font=\footnotesize] {
  \tikz\draw[arrow=traefikbg] (0,0) -- (1,0); HTTP/REST \quad
  \tikz\draw[biarrow=grpcblue, dashed] (0,0) -- (1,0); gRPC \quad
  \tikz\draw[arrow=prometheusbg, dotted] (0,0) -- (1,0); Métriques/Traces
};

\end{tikzpicture}
\caption{Topologie complète des 14 conteneurs Docker de la plateforme OnCallNova}
\label{fig:topology}
\end{figure}

% ---------- Tableau récap des services ----------
\subsection{Inventaire des 14 Conteneurs}

\begin{table}[H]
\centering
\small
\begin{tabularx}{\textwidth}{l l l X}
\toprule
\textbf{Service} & \textbf{Techno} & \textbf{Port(s)} & \textbf{Rôle} \\
\midrule
Alert Ingestion    & Python/FastAPI   & dyn / 50051  & Réception alertes, corrélation, gRPC→Incident \\
Incident Mgmt      & Python/FastAPI   & 8002 / 50052 & CRUD incidents, webhooks, analytics \\
OnCall Service     & Python/FastAPI   & 8003 / 50053 & Astreintes, rotations, escalade \\
Web UI             & Python/FastAPI   & 8080         & Dashboard temps réel, login JWT \\
Metrics Exporter   & Python/FastAPI   & 9090         & Agrégation métriques custom \\
\midrule
PostgreSQL 16      & PostgreSQL       & 5432         & Base de données relationnelle \\
Redis 7            & Redis            & 6379         & Cache, Pub/Sub, sessions \\
\midrule
Traefik v3         & Go               & 80 / 8888    & API Gateway, routage, rate limiting \\
Prometheus         & Go               & 9091         & Collecte métriques (scrape 5s) \\
Grafana 10.3       & Go               & 3000         & Visualisation dashboards \\
Jaeger             & Go               & 16686 / 4317 & Tracing distribué (OTLP) \\
Loki 3.0           & Go               & 3100         & Agrégation logs \\
Promtail           & Go               & —            & Collecte logs → Loki \\
MailHog            & Go               & 8025 / 1025  & Serveur SMTP de test local \\
\bottomrule
\end{tabularx}
\caption{Détail des 14 services Docker Compose}
\end{table}

% =============================================================================
% SECTION 3 : ARCHITECTURE DÉTAILLÉE PAR COUCHE
% =============================================================================
\section{Architecture Détaillée par Couche}

\subsection{Couche 1 — API Gateway (Traefik v3)}

\begin{figure}[H]
\centering
\begin{tikzpicture}[
  node distance=0.8cm,
  box/.style={rectangle, rounded corners=4pt, draw=traefikbg!80,
              fill=traefikbg!10, minimum width=5cm, minimum height=0.8cm,
              font=\small, align=center},
  arrow/.style={-{Stealth[length=5pt]}, thick, traefikbg!80}
]
  \node[box, fill=dark!10, draw=dark!50] (req) {Requête HTTP entrante (:80)};
  \node[box, below=of req] (tls) {Entrypoint Web};
  \node[box, below=of tls] (rate) {Middleware: Rate Limiting (100 req/s)};
  \node[box, below=of rate] (strip) {Middleware: StripPrefix};
  \node[box, below=of strip] (route) {Router: PathPrefix Matching};

  \node[box, fill=primary!10, draw=primary!60, below left=1cm and -0.3cm of route]
    (s1) {/api/v1/alerts → Alert Ingestion};
  \node[box, fill=danger!10, draw=danger!60, below=1cm of route]
    (s2) {/api/v1/incidents → Incident Mgmt};
  \node[box, fill=secondary!10, draw=secondary!60, below right=1cm and -0.3cm of route]
    (s3) {/api/v1/oncall → OnCall Service};

  \draw[arrow] (req) -- (tls);
  \draw[arrow] (tls) -- (rate);
  \draw[arrow] (rate) -- (strip);
  \draw[arrow] (strip) -- (route);
  \draw[arrow] (route) -| (s1);
  \draw[arrow] (route) -- (s2);
  \draw[arrow] (route) -| (s3);
\end{tikzpicture}
\caption{Pipeline de traitement Traefik — du client au microservice}
\label{fig:traefik}
\end{figure}

\begin{justifbox}{Justification : Traefik v3}
\begin{itemize}[nosep]
  \item \textbf{Découverte automatique} via labels Docker — zéro configuration de routes manuelles.
  \item \textbf{Rate limiting natif} (100 req/s par IP) sans code applicatif.
  \item \textbf{Dashboard intégré} (:8888) pour visualiser les routes en temps réel.
  \item \textbf{Support scaling} : découvre automatiquement les nouvelles instances
        lors d'un \tech{docker compose up --scale alert-ingestion=3}.
  \item \textbf{Alternatives rejetées} : Nginx (configuration statique, pas de découverte Docker),
        Kong (trop lourd pour un déploiement 100\% local).
\end{itemize}
\end{justifbox}

% ---------- Couche 2 : Microservices ----------
\subsection{Couche 2 — Les 5 Microservices}

\subsubsection{Alert Ingestion Service}

\begin{figure}[H]
\centering
\begin{tikzpicture}[
  node distance=0.6cm and 1.5cm,
  process/.style={rectangle, rounded corners=4pt, draw=primary!70,
                  fill=primary!8, minimum width=3.5cm, minimum height=0.9cm,
                  font=\small, align=center},
  data/.style={cylinder, draw=postgresbg!70, fill=postgresbg!8,
               minimum width=1.8cm, minimum height=0.8cm, font=\small,
               shape border rotate=90, aspect=0.3},
  arrow/.style={-{Stealth[length=5pt]}, thick}
]
  \node[process] (recv) {\faIcon{inbox}~Réception Alerte\\POST /api/v1/alerts};
  \node[process, below=of recv] (valid) {\faIcon{check}~Validation Pydantic\\source, severity, message};
  \node[process, below=of valid] (dedup) {\faIcon{clone}~Déduplication\\Redis fingerprint (SHA-256)};
  \node[process, below=of dedup] (corr) {\faIcon{project-diagram}~Corrélation\\Fenêtre 5 min + source};
  \node[process, below left=1cm and 0cm of corr] (create) {\faIcon{plus-circle}~Nouveau Incident\\gRPC → Incident Mgmt};
  \node[process, below right=1cm and 0cm of corr] (attach) {\faIcon{paperclip}~Rattachement\\Incident existant};

  \draw[arrow] (recv) -- (valid);
  \draw[arrow] (valid) -- (dedup);
  \draw[arrow] (dedup) -- (corr);
  \draw[arrow] (corr) -- node[left, font=\scriptsize] {Pas de match} (create);
  \draw[arrow] (corr) -- node[right, font=\scriptsize] {Match trouvé} (attach);

  % Annotation algorithme
  \node[right=2.5cm of corr, text width=4.5cm, font=\scriptsize, align=left,
        draw=primary!40, fill=primary!5, rounded corners=3pt, inner sep=6pt]
    (algo) {
      \textbf{Algorithme de corrélation:}\\[3pt]
      1. Hash = SHA-256(source + severity)\\
      2. Recherche dans Redis: clé = \texttt{corr:\{hash\}}\\
      3. Si trouvé et $\Delta t < 300$s → rattacher\\
      4. Sinon → créer incident + stocker clé Redis avec TTL=300s
    };
  \draw[dashed, primary!40] (corr) -- (algo);
\end{tikzpicture}
\caption{Pipeline de traitement du service Alert Ingestion}
\label{fig:alert-pipeline}
\end{figure}

\begin{justifbox}{Justification : Corrélation par fenêtre temporelle}
\begin{itemize}[nosep]
  \item Réduit le \textbf{bruit d'alertes} : 100 alertes similaires en 5 minutes
        = 1 seul incident.
  \item \textbf{Redis comme cache de corrélation} : TTL natif pour l'expiration
        automatique de la fenêtre — $O(1)$ en lecture/écriture.
  \item \textbf{SHA-256 fingerprint} : collision quasi-nulle, déduplication fiable.
  \item Fenêtre de 5 minutes configurable via variable d'environnement
        \tech{CORRELATION\_WINDOW}.
\end{itemize}
\end{justifbox}

\subsubsection{Incident Management Service}

\begin{figure}[H]
\centering
\begin{tikzpicture}[
  state/.style={circle, draw=#1!80, fill=#1!15, minimum size=1.6cm,
                font=\small\bfseries, text=#1!80},
  arrow/.style={-{Stealth[length=6pt]}, thick, dark!60},
  label/.style={font=\scriptsize, midway, above, sloped}
]
  \node[state=danger] (firing) {FIRING};
  \node[state=accent, right=3cm of firing] (ack) {ACK};
  \node[state=secondary, right=3cm of ack] (resolved) {RESOLVED};
  \node[state=primary, below=2cm of ack] (escalated) {ESCALATED};

  \draw[arrow] (firing) -- node[label] {acknowledge()} (ack);
  \draw[arrow] (ack) -- node[label] {resolve()} (resolved);
  \draw[arrow] (firing) -- node[label, below, sloped] {timeout 30 min} (escalated);
  \draw[arrow] (escalated) -- node[label] {acknowledge()} (ack);
  \draw[arrow, bend right=20] (escalated.east) -- node[label, below] {resolve()} (resolved.south);

  % Annotations
  \node[below=0.3cm of firing, font=\scriptsize\color{dark!50}] {Création};
  \node[below=0.3cm of ack, font=\scriptsize\color{dark!50}] {Prise en charge};
  \node[below=0.3cm of resolved, font=\scriptsize\color{dark!50}] {Clôture};
  \node[below=0.3cm of escalated, font=\scriptsize\color{dark!50}] {Niveau supérieur};
\end{tikzpicture}
\caption{Machine à états du cycle de vie d'un incident}
\label{fig:incident-states}
\end{figure}

\begin{techbox}{Fonctionnalités clés — Incident Management}
\begin{multicols}{2}
\begin{itemize}[nosep]
  \item CRUD complet (REST + gRPC)
  \item Machine à états FIRING → ACK → RESOLVED
  \item Escalade automatique (timeout 30 min)
  \item Historique complet (timeline)
  \item Calcul MTTA / MTTR automatique
  \item Notifications email (MailHog)
  \item Dispatch webhooks (HMAC-SHA256)
  \item Analytics (trends \& MTTR distribution)
  \item Protection JWT sur endpoints d'écriture
  \item Métriques Prometheus exposées
\end{itemize}
\end{multicols}
\end{techbox}

\subsubsection{OnCall Service}

\begin{figure}[H]
\centering
\begin{tikzpicture}[
  node distance=0.7cm,
  box/.style={rectangle, rounded corners=4pt, draw=secondary!70,
              fill=secondary!8, minimum width=4.5cm, minimum height=0.8cm,
              font=\small, align=center},
  arrow/.style={-{Stealth[length=5pt]}, thick, secondary!60}
]
  \node[box] (teams) {\faIcon{users}~Gestion des Équipes\\CRUD /api/v1/teams};
  \node[box, below=of teams] (sched) {\faIcon{calendar-alt}~Planification Rotations\\schedules avec start/end};
  \node[box, below=of sched] (oncall) {\faIcon{user-shield}~Identification On-Call\\GET /api/v1/oncall/current};
  \node[box, below=of oncall] (esc) {\faIcon{level-up-alt}~Escalade Niveau N\\gRPC GetCurrentOnCall};

  \draw[arrow] (teams) -- (sched);
  \draw[arrow] (sched) -- (oncall);
  \draw[arrow] (oncall) -- (esc);

  \node[right=2cm of sched, text width=4cm, font=\scriptsize,
        draw=secondary!40, fill=secondary!5, rounded corners=3pt, inner sep=6pt]
    (rot) {
      \textbf{Algorithme de rotation:}\\[3pt]
      \texttt{now = datetime.utcnow()}\\
      \texttt{rotation = schedules}\\
      \texttt{~~.filter(start <= now)}\\
      \texttt{~~.filter(end > now)}\\
      \texttt{~~.order\_by(level)}\\
      \texttt{~~.first()}
    };
  \draw[dashed, secondary!40] (sched) -- (rot);
\end{tikzpicture}
\caption{Fonctionnement du service d'astreintes OnCall}
\label{fig:oncall}
\end{figure}

\subsubsection{Web UI}

\begin{figure}[H]
\centering
\begin{tikzpicture}[
  node distance=0.6cm and 2cm,
  page/.style={rectangle, rounded corners=4pt, draw=accent!70,
               fill=accent!8, minimum width=3cm, minimum height=0.8cm,
               font=\small, align=center},
  arrow/.style={-{Stealth[length=5pt]}, thick, accent!50}
]
  \node[page] (login) {\faIcon{lock}~Login\\JWT Auth};
  \node[page, right=of login] (dash) {\faIcon{tachometer-alt}~Dashboard\\Vue temps réel};
  \node[page, above right=0.5cm and 2cm of dash] (inc) {\faIcon{fire}~Incidents\\Table + Actions};
  \node[page, right=2cm of dash] (teams) {\faIcon{users}~Teams\\Gestion équipes};
  \node[page, below right=0.5cm and 2cm of dash] (oncall) {\faIcon{user-clock}~On-Call\\Rotations};

  \draw[arrow] (login) -- node[above, font=\scriptsize] {Cookie JWT} (dash);
  \draw[arrow] (dash) -- (inc);
  \draw[arrow] (dash) -- (teams);
  \draw[arrow] (dash) -- (oncall);

  \node[below=1cm of dash, font=\scriptsize, text width=6cm, align=center,
        draw=accent!30, rounded corners=3pt, fill=accent!3, inner sep=5pt] {
    \textbf{Stack Frontend:} Jinja2 Templates + TailwindCSS CDN\\
    Auto-refresh toutes les 30s | Responsive Design
  };
\end{tikzpicture}
\caption{Navigation de l'interface Web UI}
\label{fig:webui}
\end{figure}

\subsubsection{Metrics Exporter}

\begin{techbox}{Métriques exposées (/metrics)}
\begin{itemize}[nosep]
  \item \tech{oncall\_incidents\_total} — Compteur total d'incidents (par sévérité)
  \item \tech{oncall\_incidents\_active} — Jauge d'incidents actifs
  \item \tech{oncall\_mtta\_seconds} — Histogramme Mean Time To Acknowledge
  \item \tech{oncall\_mttr\_seconds} — Histogramme Mean Time To Resolve
  \item \tech{oncall\_alerts\_ingested\_total} — Compteur d'alertes ingérées
  \item \tech{oncall\_oncall\_engineers\_active} — Jauge d'ingénieurs de garde
  \item Format : Prometheus text exposition (scrape interval : 5s)
\end{itemize}
\end{techbox}

% ---------- Couche 3 : Communication ----------
\subsection{Couche 3 — Communication Inter-Services}

\begin{figure}[H]
\centering
\begin{tikzpicture}[
  node distance=1.5cm,
  svc/.style={rectangle, rounded corners=6pt, draw=#1!70, fill=#1!10,
              minimum width=3cm, minimum height=1.2cm, font=\small\bfseries,
              align=center},
  proto/.style={rectangle, draw=grpcblue!60, fill=grpcblue!8,
                rounded corners=3pt, font=\scriptsize, inner sep=4pt}
]
  \node[svc=primary] (a) {Alert\\Ingestion};
  \node[svc=danger, right=4cm of a] (i) {Incident\\Management};
  \node[svc=secondary, right=4cm of i] (o) {OnCall\\Service};

  % gRPC calls
  \draw[-{Stealth[length=6pt]}, thick, grpcblue, dashed]
    (a) -- node[above, font=\scriptsize\bfseries\color{grpcblue}] {gRPC} (i);
  \draw[-{Stealth[length=6pt]}, thick, grpcblue, dashed]
    (i) -- node[above, font=\scriptsize\bfseries\color{grpcblue}] {gRPC} (o);

  % Proto definitions
  \node[proto, below=0.8cm of a] (p1) {CreateIncident()\\UpdateIncident()};
  \node[proto, below=0.8cm of i] (p2) {GetIncident()\\ListIncidents()};
  \node[proto, below=0.8cm of o] (p3) {GetCurrentOnCall()\\ListTeams()};

  \draw[dotted, grpcblue!50] (a) -- (p1);
  \draw[dotted, grpcblue!50] (i) -- (p2);
  \draw[dotted, grpcblue!50] (o) -- (p3);

  % Redis pub/sub
  \node[rectangle, rounded corners=4pt, draw=redisbg!70, fill=redisbg!10,
        minimum width=2.5cm, minimum height=0.8cm, font=\small\bfseries,
        below=3cm of i] (redis) {\color{redisbg}Redis Pub/Sub};

  \draw[-{Stealth[length=5pt]}, redisbg!60] (a.south) -- ++(0,-0.5) -| (redis)
    node[near start, right, font=\scriptsize] {PUBLISH alert:new};
  \draw[-{Stealth[length=5pt]}, redisbg!60] (redis) -| (i.south)
    node[near end, left, font=\scriptsize] {SUBSCRIBE};

\end{tikzpicture}
\caption{Communication inter-services : gRPC synchrone + Redis Pub/Sub asynchrone}
\label{fig:comms}
\end{figure}

\begin{justifbox}{Justification : gRPC + Protobuf}
\begin{itemize}[nosep]
  \item \textbf{Performance} : sérialisation binaire Protobuf 5-10$\times$ plus rapide que JSON.
  \item \textbf{Contrat fort} : fichiers \tech{.proto} = documentation + validation automatique.
  \item \textbf{Streaming} : support natif du streaming bidirectionnel pour les futures évolutions.
  \item \textbf{Code generation} : stubs Python générés automatiquement (\tech{grpcio-tools}).
  \item \textbf{Alternatives rejetées} : REST interne (overhead HTTP, pas de contrat),
        RabbitMQ (complexité supplémentaire, Redis suffit pour le Pub/Sub simple).
\end{itemize}
\end{justifbox}

% ---------- Couche Data ----------
\subsection{Couche 4 — Persistance des Données}

\begin{figure}[H]
\centering
\begin{tikzpicture}[
  node distance=0.8cm,
  table/.style={rectangle, draw=postgresbg!70, fill=postgresbg!8,
                minimum width=3.8cm, minimum height=0.7cm, font=\small},
  cache/.style={rectangle, draw=redisbg!70, fill=redisbg!8,
                minimum width=3.8cm, minimum height=0.7cm, font=\small},
  title/.style={font=\small\bfseries, color=#1}
]
  % PostgreSQL
  \node[title=postgresbg] (pgtitle) {\faIcon{database}~PostgreSQL 16 — Source of Truth};
  \node[table, below=0.3cm of pgtitle] (t1) {\texttt{alerts} — Alertes brutes};
  \node[table, below=0.15cm of t1] (t2) {\texttt{incidents} — Incidents + état};
  \node[table, below=0.15cm of t2] (t3) {\texttt{incident\_alerts} — Liaison N:M};
  \node[table, below=0.15cm of t3] (t4) {\texttt{teams} — Équipes};
  \node[table, below=0.15cm of t4] (t5) {\texttt{team\_members} — Membres};
  \node[table, below=0.15cm of t5] (t6) {\texttt{schedules} — Rotations};

  % Redis
  \node[title=redisbg, right=3cm of pgtitle] (rdtitle) {\faIcon{bolt}~Redis 7 — Cache \& Pub/Sub};
  \node[cache, below=0.3cm of rdtitle] (r1) {\texttt{corr:\{hash\}} — Corrélation (TTL 5m)};
  \node[cache, below=0.15cm of r1] (r2) {\texttt{dedup:\{fp\}} — Déduplication};
  \node[cache, below=0.15cm of r2] (r3) {\texttt{alert:new} — Canal Pub/Sub};
  \node[cache, below=0.15cm of r3] (r4) {\texttt{webhooks} — Config webhooks};
  \node[cache, below=0.15cm of r4] (r5) {\texttt{metrics:*} — Cache métriques};

  % Boxes
  \node[draw=postgresbg!40, rounded corners=6pt, fit=(pgtitle)(t6),
        inner sep=8pt] {};
  \node[draw=redisbg!40, rounded corners=6pt, fit=(rdtitle)(r5),
        inner sep=8pt] {};

\end{tikzpicture}
\caption{Modèle de données — PostgreSQL (persistant) + Redis (éphémère)}
\label{fig:data}
\end{figure}

\begin{justifbox}{Justification : PostgreSQL + Redis}
\begin{itemize}[nosep]
  \item \textbf{PostgreSQL} : ACID complet, JSONB pour métadonnées flexibles,
        requêtes analytiques performantes (trends, MTTR).
  \item \textbf{SQLAlchemy Async 2.0} : ORM avec support \tech{asyncpg} —
        performances optimales sans bloquer l'event loop FastAPI.
  \item \textbf{Redis} : latence sub-milliseconde pour la corrélation en temps réel,
        TTL natif pour l'expiration des fenêtres, Pub/Sub intégré.
  \item \textbf{Séparation des responsabilités} : PostgreSQL = source de vérité,
        Redis = cache transitoire.
\end{itemize}
\end{justifbox}

% =============================================================================
% SECTION 4 : SÉCURITÉ
% =============================================================================
\section{Sécurité}

\subsection{Architecture d'Authentification JWT}

\begin{figure}[H]
\centering
\begin{tikzpicture}[
  node distance=0.8cm and 2cm,
  step/.style={rectangle, rounded corners=4pt, draw=danger!60,
               fill=danger!8, minimum width=4.5cm, minimum height=0.8cm,
               font=\small, align=center},
  arrow/.style={-{Stealth[length=5pt]}, thick, danger!50}
]
  \node[step] (login) {1. POST /login\\username + password};
  \node[step, below=of login] (verify) {2. authenticate\_user()\\Vérification credentials (.env)};
  \node[step, below=of verify] (jwt) {3. create\_jwt\_token()\\HMAC-SHA256, exp=24h};
  \node[step, below=of jwt] (cookie) {4. Set-Cookie: token=\{jwt\}\\HttpOnly, Path=/};
  \node[step, below=of cookie] (mid) {5. Middleware Auth\\Vérifie JWT sur chaque requête};
  \node[step, below=of mid] (access) {6. Accès autorisé\\user injecté dans le contexte};

  \draw[arrow] (login) -- (verify);
  \draw[arrow] (verify) -- (jwt);
  \draw[arrow] (jwt) -- (cookie);
  \draw[arrow] (cookie) -- (mid);
  \draw[arrow] (mid) -- (access);

  \node[right=2cm of jwt, text width=5cm, font=\scriptsize,
        draw=danger!30, fill=danger!3, rounded corners=3pt, inner sep=6pt] (payload) {
    \textbf{JWT Payload:}\\[3pt]
    \texttt{\{}\\
    \texttt{~~"sub": "admin",}\\
    \texttt{~~"role": "admin",}\\
    \texttt{~~"exp": 1738454400,}\\
    \texttt{~~"iat": 1738368000}\\
    \texttt{\}}\\[5pt]
    \textbf{Algorithme:} HS256\\
    \textbf{Secret:} via \$JWT\_SECRET
  };
  \draw[dashed, danger!30] (jwt) -- (payload);
\end{tikzpicture}
\caption{Flux d'authentification JWT — du login à l'accès protégé}
\label{fig:jwt}
\end{figure}

\begin{warnbox}{Mesures de Sécurité}
\begin{multicols}{2}
\begin{itemize}[nosep]
  \item[\cmark] JWT HMAC-SHA256
  \item[\cmark] Expiration 24h configurable
  \item[\cmark] Secrets via variables d'env
  \item[\cmark] Rate limiting Traefik (100/s)
  \item[\cmark] Cookie HttpOnly
  \item[\cmark] Pas de credentials en dur
  \item[\cmark] Endpoints sensibles protégés
  \item[\cmark] Bearer token pour API
  \item[\cmark] Middleware global Web UI
  \item[\cmark] Support multi-utilisateurs
\end{itemize}
\end{multicols}
\end{warnbox}

% =============================================================================
% SECTION 5 : OBSERVABILITÉ
% =============================================================================
\section{Stack d'Observabilité}

\subsection{Les 3 Piliers}

\begin{figure}[H]
\centering
\begin{tikzpicture}[
  pillar/.style={rectangle, rounded corners=8pt, draw=#1!70, fill=#1!10,
                 minimum width=4.5cm, minimum height=3cm, font=\small, align=center},
  title/.style={font=\bfseries\color{#1}}
]
  \node[pillar=prometheusbg] (metrics) {
    \title{prometheusbg}{Métriques}\\[5pt]
    Prometheus\\scrape 5s\\[3pt]
    Grafana dashboards\\MTTA / MTTR / Trends
  };
  \node[pillar=jaegerbg, right=1cm of metrics] (traces) {
    \title{jaegerbg}{Traces}\\[5pt]
    Jaeger\\OpenTelemetry\\[3pt]
    Tracing distribué\\Latence end-to-end
  };
  \node[pillar=lokibg, right=1cm of traces] (logs) {
    \title{lokibg}{Logs}\\[5pt]
    Loki + Promtail\\Agrégation centralisée\\[3pt]
    Labels par service\\Requêtes LogQL
  };

  % Grafana unificateur
  \node[rectangle, rounded corners=6pt, draw=grafanabg!80, fill=grafanabg!15,
        minimum width=14.5cm, minimum height=1cm, font=\bfseries,
        below=1cm of traces] (grafana)
    {\faIcon{chart-area}~Grafana 10.3 — Vue Unifiée (Métriques + Traces + Logs)};

  \draw[-{Stealth[length=5pt]}, thick, prometheusbg!60] (metrics) -- (grafana);
  \draw[-{Stealth[length=5pt]}, thick, jaegerbg!60] (traces) -- (grafana);
  \draw[-{Stealth[length=5pt]}, thick, lokibg!60] (logs) -- (grafana);
\end{tikzpicture}
\caption{Les 3 piliers de l'observabilité unifiés dans Grafana}
\label{fig:observability}
\end{figure}

\begin{justifbox}{Justification : Stack LGTM (Loki, Grafana, Tempo→Jaeger, Mimir→Prometheus)}
\begin{itemize}[nosep]
  \item \textbf{Standard industriel} : stack Grafana Labs / CNCF, utilisée par les leaders du marché.
  \item \textbf{OpenTelemetry} : standard vendeur-neutre pour l'instrumentation — un seul SDK pour
        métriques, traces et logs.
  \item \textbf{Corrélation} : Grafana permet de naviguer d'une métrique → une trace → des logs
        d'un seul clic.
  \item \textbf{Alternatives rejetées} : ELK Stack (lourd, license), Datadog (SaaS payant),
        New Relic (pas 100\% local).
\end{itemize}
\end{justifbox}

% =============================================================================
% SECTION 6 : FONCTIONNALITÉS BONUS
% =============================================================================
\section{Fonctionnalités Bonus}

\subsection{Vue d'ensemble}

\begin{table}[H]
\centering
\begin{tabularx}{\textwidth}{c l l c X}
\toprule
\textbf{\#} & \textbf{Bonus} & \textbf{Technologie} & \textbf{Statut} & \textbf{Description} \\
\midrule
1 & Notifications Email    & MailHog / SMTP      & \cmark & Envoi d'emails lors des changements d'état \\
2 & Webhooks               & HMAC-SHA256         & \cmark & Dispatch sécurisé vers URLs externes \\
3 & Escalade Automatique   & Cron + gRPC         & \cmark & Escalade après timeout (30 min) \\
4 & Analytics Historiques  & SQL + agrégations   & \cmark & Trends, MTTR distribution, rapports \\
5 & Agrégation de Logs     & Loki 3.0 + Promtail & $\sim$ & Code présent, limité sur Windows Docker \\
6 & Tracing Distribué      & Jaeger + OTLP       & \cmark & Traces end-to-end inter-services \\
7 & Scaling Horizontal     & Docker Compose      & \cmark & \texttt{--scale alert-ingestion=3} \\
\bottomrule
\end{tabularx}
\caption{Tableau récapitulatif des 7 fonctionnalités bonus}
\end{table}

% ---------- Bonus 1 : Email ----------
\subsection{Bonus 1 — Notifications Email}

\begin{figure}[H]
\centering
\begin{tikzpicture}[
  node distance=1cm and 2cm,
  box/.style={rectangle, rounded corners=4pt, draw=primary!60, fill=primary!8,
              minimum width=3cm, minimum height=0.8cm, font=\small, align=center},
  arrow/.style={-{Stealth[length=5pt]}, thick, primary!50}
]
  \node[box] (event) {Événement\\(create/ack/resolve)};
  \node[box, right=of event] (build) {build\_incident\_email()\\Sujet + Corps HTML};
  \node[box, right=of build] (send) {send\_notification\_email()\\SMTP :1025};
  \node[box, right=of send, draw=secondary!60, fill=secondary!8]
    (mailhog) {\faIcon{envelope}~MailHog\\UI :8025};

  \draw[arrow] (event) -- (build);
  \draw[arrow] (build) -- (send);
  \draw[arrow] (send) -- (mailhog);
\end{tikzpicture}
\caption{Pipeline de notification email via MailHog}
\end{figure}

\begin{techbox}{Détails techniques — Email}
\begin{itemize}[nosep]
  \item \textbf{Transport} : SMTP standard (port 1025), compatible SendGrid en production.
  \item \textbf{MailHog} : capture tous les emails localement sans envoi réel — parfait pour le développement.
  \item \textbf{Templates} : emails HTML formatés avec sévérité, titre, timestamp.
  \item \textbf{Configuration} : \tech{SMTP\_HOST}, \tech{SMTP\_PORT}, \tech{SMTP\_FROM} via \texttt{.env}.
\end{itemize}
\end{techbox}

% ---------- Bonus 2 : Webhooks ----------
\subsection{Bonus 2 — Webhooks Sécurisés}

\begin{figure}[H]
\centering
\begin{tikzpicture}[
  node distance=0.7cm and 1.5cm,
  box/.style={rectangle, rounded corners=4pt, draw=accent!60, fill=accent!8,
              minimum width=3.5cm, minimum height=0.8cm, font=\small, align=center},
  arrow/.style={-{Stealth[length=5pt]}, thick, accent!50}
]
  \node[box] (event) {Incident créé/modifié};
  \node[box, below=of event] (load) {load\_webhooks\_from\_redis()};
  \node[box, below=of load] (sign) {HMAC-SHA256 Signature\\X-Webhook-Signature};
  \node[box, below=of sign] (dispatch) {POST → URL cible\\httpx.AsyncClient};

  \draw[arrow] (event) -- (load);
  \draw[arrow] (load) -- (sign);
  \draw[arrow] (sign) -- (dispatch);

  \node[right=2cm of sign, text width=5cm, font=\scriptsize,
        draw=accent!30, fill=accent!3, rounded corners=3pt, inner sep=6pt] {
    \textbf{Payload webhook:}\\[3pt]
    \texttt{\{event, incident\_id,}\\
    \texttt{~~title, severity, status,}\\
    \texttt{~~timestamp, source\}}\\[5pt]
    \textbf{Header:}\\
    \texttt{X-Webhook-Signature:}\\
    \texttt{sha256=\{hmac\}}
  };
\end{tikzpicture}
\caption{Architecture du système de webhooks}
\end{figure}

% ---------- Bonus 3 : Escalade ----------
\subsection{Bonus 3 — Escalade Automatique}

\begin{figure}[H]
\centering
\begin{tikzpicture}[
  node distance=1cm,
  step/.style={rectangle, rounded corners=4pt, draw=danger!60, fill=danger!8,
               minimum width=5cm, minimum height=0.8cm, font=\small, align=center},
  arrow/.style={-{Stealth[length=5pt]}, thick, danger!50}
]
  \node[step] (detect) {\faIcon{clock}~Détection timeout\\Incident FIRING > 30 min sans ACK};
  \node[step, below=of detect] (grpc) {\faIcon{phone}~gRPC → OnCall Service\\GetCurrentOnCall(level=N+1)};
  \node[step, below=of grpc] (notify) {\faIcon{bell}~Notification\\Email + Webhook au nouveau responsable};
  \node[step, below=of notify] (update) {\faIcon{edit}~Mise à jour\\status=ESCALATED, assignee=nouveau};

  \draw[arrow] (detect) -- (grpc);
  \draw[arrow] (grpc) -- (notify);
  \draw[arrow] (notify) -- (update);
\end{tikzpicture}
\caption{Processus d'escalade automatique multi-niveaux}
\end{figure}

% ---------- Bonus 4 : Analytics ----------
\subsection{Bonus 4 — Analytics Historiques}

\begin{techbox}{Endpoints Analytics}
\begin{itemize}[nosep]
  \item \textbf{GET /api/v1/analytics/trends}
  \begin{itemize}[nosep]
    \item Paramètres : \texttt{days} (défaut 30), \texttt{severity}
    \item Retourne : incidents par jour, par sévérité, totaux
    \item Requête SQL avec \texttt{GROUP BY date\_trunc('day', created\_at)}
  \end{itemize}
  \item \textbf{GET /api/v1/analytics/mttr-distribution}
  \begin{itemize}[nosep]
    \item Distribution des temps de résolution
    \item Calcul des percentiles P50, P90, P99
    \item Segmentation par sévérité (critical, high, medium, low)
  \end{itemize}
\end{itemize}
\end{techbox}

% ---------- Bonus 5 : Loki ----------
\subsection{Bonus 5 — Agrégation de Logs (Loki)}

\begin{techbox}{Architecture Loki}
\begin{itemize}[nosep]
  \item \textbf{Loki 3.0} : stockage et indexation des logs par labels (pas full-text).
  \item \textbf{Promtail} : agent de collecte qui scrape les logs Docker.
  \item \textbf{LokiHandler} (Python) : handler \tech{logging.Handler} custom qui push
        directement vers l'API HTTP Loki (\texttt{/loki/api/v1/push}).
  \item \textbf{Labels} : \texttt{service}, \texttt{level}, \texttt{environment}.
  \item \textbf{Note} : fonctionnel sur Linux, limité sur Windows Docker Desktop
        (incompatibilité API Docker socket pour Promtail).
\end{itemize}
\end{techbox}

% ---------- Bonus 6 : Jaeger ----------
\subsection{Bonus 6 — Tracing Distribué (Jaeger)}

\begin{figure}[H]
\centering
\begin{tikzpicture}[
  node distance=0.5cm and 2.5cm,
  svc/.style={rectangle, rounded corners=6pt, draw=jaegerbg!70, fill=jaegerbg!10,
              minimum width=2.8cm, minimum height=1cm, font=\small\bfseries,
              align=center},
  arrow/.style={-{Stealth[length=5pt]}, thick, jaegerbg!60}
]
  \node[svc] (client) {Client HTTP};
  \node[svc, right=of client] (alert) {Alert\\Ingestion};
  \node[svc, right=of alert] (incident) {Incident\\Management};
  \node[svc, right=of incident] (oncall) {OnCall\\Service};

  \draw[arrow] (client) -- node[above, font=\scriptsize] {trace-id} (alert);
  \draw[arrow] (alert) -- node[above, font=\scriptsize] {span} (incident);
  \draw[arrow] (incident) -- node[above, font=\scriptsize] {span} (oncall);

  % Jaeger collector
  \node[rectangle, rounded corners=6pt, draw=jaegerbg!80, fill=jaegerbg!20,
        minimum width=10cm, minimum height=1cm, font=\small\bfseries,
        below=1.5cm of incident] (jaeger)
    {\faIcon{search}~Jaeger Collector — OTLP gRPC :4317 / HTTP :4318};

  \draw[arrow, dotted] (alert.south) -- (jaeger);
  \draw[arrow, dotted] (incident.south) -- (jaeger);
  \draw[arrow, dotted] (oncall.south) -- (jaeger);

  % Annotation
  \node[below=0.3cm of jaeger, font=\scriptsize\color{dark!50}] {
    UI : http://localhost:16686 — Visualisation des traces \& spans
  };
\end{tikzpicture}
\caption{Propagation des traces distribuées via OpenTelemetry → Jaeger}
\end{figure}

\begin{justifbox}{Justification : OpenTelemetry + Jaeger}
\begin{itemize}[nosep]
  \item \textbf{OpenTelemetry} : standard CNCF vendeur-neutre — instrumentation une seule fois,
        exportation vers n'importe quel backend (Jaeger, Zipkin, Datadog\ldots).
  \item \textbf{Propagation automatique} : le \tech{trace-id} est propagé dans les headers HTTP
        et métadonnées gRPC entre tous les services.
  \item \textbf{Jaeger All-in-One} : déploiement simplifié (collector + query + UI en 1 conteneur).
  \item \textbf{Impact performances} : overhead < 3\% grâce au sampling.
\end{itemize}
\end{justifbox}

% ---------- Bonus 7 : Scaling ----------
\subsection{Bonus 7 — Scaling Horizontal}

\begin{figure}[H]
\centering
\begin{tikzpicture}[
  node distance=0.5cm and 1cm,
  instance/.style={rectangle, rounded corners=4pt, draw=primary!60, fill=primary!8,
                   minimum width=2.2cm, minimum height=0.8cm, font=\small, align=center},
  lb/.style={rectangle, rounded corners=6pt, draw=traefikbg!80, fill=traefikbg!15,
             minimum width=3.5cm, minimum height=1.2cm, font=\small\bfseries,
             align=center}
]
  \node[lb] (traefik) {\faIcon{balance-scale}~Traefik\\Load Balancer};

  \node[instance, below left=1.5cm and 1cm of traefik] (i1) {Instance 1\\:8001 (dyn)};
  \node[instance, below=1.5cm of traefik] (i2) {Instance 2\\:8001 (dyn)};
  \node[instance, below right=1.5cm and 1cm of traefik] (i3) {Instance 3\\:8001 (dyn)};

  \draw[-{Stealth[length=5pt]}, thick, traefikbg!60] (traefik) -- (i1);
  \draw[-{Stealth[length=5pt]}, thick, traefikbg!60] (traefik) -- (i2);
  \draw[-{Stealth[length=5pt]}, thick, traefikbg!60] (traefik) -- (i3);

  % Labels
  \node[above=0.3cm of traefik, font=\scriptsize\color{dark!50}] {
    docker compose up --scale alert-ingestion=3
  };
  \node[below=0.3cm of i2, font=\scriptsize\color{dark!50}, text width=8cm, align=center] {
    Pas de port hôte fixe — Traefik découvre automatiquement\\
    les instances via les labels Docker et répartit la charge (round-robin)
  };
\end{tikzpicture}
\caption{Scaling horizontal du service Alert Ingestion via Docker Compose}
\end{figure}

\begin{justifbox}{Justification : Scaling via Docker Compose + Traefik}
\begin{itemize}[nosep]
  \item \textbf{Zéro configuration supplémentaire} : Traefik découvre les conteneurs
        automatiquement via l'API Docker.
  \item \textbf{Port dynamique} : le service n'a pas de port hôte fixe, évitant les
        conflits lors du scaling.
  \item \textbf{Load balancing} : round-robin natif entre les instances.
  \item \textbf{Commande unique} : \texttt{docker compose up -d --scale alert-ingestion=3}.
\end{itemize}
\end{justifbox}

% =============================================================================
% SECTION 7 : CHOIX TECHNIQUES & JUSTIFICATIONS
% =============================================================================
\section{Synthèse des Choix Techniques}

\begin{table}[H]
\centering
\small
\begin{tabularx}{\textwidth}{l l X X}
\toprule
\textbf{Composant} & \textbf{Choix} & \textbf{Justification} & \textbf{Alternative rejetée} \\
\midrule
Langage        & Python 3.11       & Écosystème riche, async natif, prototypage rapide
                                   & Go (temps de dev), Java (verbosité) \\
Framework API  & FastAPI 0.109     & Async natif, validation Pydantic, OpenAPI auto
                                   & Flask (sync), Django (monolithique) \\
Communication  & gRPC + Protobuf   & Contrat fort, perf binaire, génération de code
                                   & REST interne (pas de contrat) \\
BDD            & PostgreSQL 16     & ACID, JSONB, performances analytiques
                                   & MongoDB (pas ACID), MySQL (moins de features) \\
Cache          & Redis 7           & Sub-ms latence, TTL natif, Pub/Sub intégré
                                   & Memcached (pas de Pub/Sub) \\
ORM            & SQLAlchemy 2.0    & Async (asyncpg), migrations, mature
                                   & Tortoise (moins mature), raw SQL \\
Gateway        & Traefik v3        & Auto-discovery Docker, rate limiting, dashboard
                                   & Nginx (config statique), Kong (lourd) \\
Conteneurs     & Docker Compose    & Orchestration simple, 100\% local, reproductible
                                   & Kubernetes (overkill pour local) \\
Métriques      & Prometheus        & Pull model, PromQL puissant, standard k8s
                                   & InfluxDB (push model), StatsD \\
Dashboards     & Grafana 10.3      & Multi-datasource, alerting, communauté
                                   & Kibana (ELK only) \\
Tracing        & Jaeger + OTel     & CNCF, vendeur-neutre, all-in-one
                                   & Zipkin (moins de features), X-Ray (AWS only) \\
Auth           & JWT HS256         & Stateless, standard, simple à implémenter
                                   & OAuth2 (complexe pour local), sessions \\
Email test     & MailHog           & Capture locale, UI web, zéro config
                                   & Mailtrap (SaaS), Papercut \\
\bottomrule
\end{tabularx}
\caption{Matrice décisionnelle des choix technologiques}
\end{table}

% =============================================================================
% SECTION 8 : DÉPLOIEMENT
% =============================================================================
\section{Déploiement \& Utilisation}

\subsection{Architecture de Build}

\begin{figure}[H]
\centering
\begin{tikzpicture}[
  node distance=0.7cm,
  stage/.style={rectangle, rounded corners=4pt, draw=dark!60, fill=dark!8,
                minimum width=6cm, minimum height=0.8cm, font=\small, align=center},
  arrow/.style={-{Stealth[length=5pt]}, thick, dark!40}
]
  \node[stage, fill=primary!8, draw=primary!60] (s1) {\textbf{Stage 1 — Builder}\\python:3.11-alpine + protoc + grpcio-tools};
  \node[stage, below=of s1] (s2) {Compilation Protobuf\\*.proto → *\_pb2.py + *\_pb2\_grpc.py};
  \node[stage, below=of s2] (s3) {Installation dépendances\\pip install -r requirements.txt};
  \node[stage, fill=secondary!8, draw=secondary!60, below=of s3]
    (s4) {\textbf{Stage 2 — Runtime}\\Image finale allégée (< 150 MB)};

  \draw[arrow] (s1) -- (s2);
  \draw[arrow] (s2) -- (s3);
  \draw[arrow] (s3) -- (s4);

  \node[right=1.5cm of s1, text width=4cm, font=\scriptsize, align=left,
        draw=primary!30, fill=primary!3, rounded corners=3pt, inner sep=6pt] {
    \textbf{Multi-stage build:}\\
    Sépare les outils de\\compilation de l'image\\
    finale pour réduire\\la surface d'attaque\\et la taille de l'image.
  };
\end{tikzpicture}
\caption{Dockerfile multi-stage pour les microservices}
\end{figure}

\subsection{Commandes de Déploiement}

\begin{techbox}{Lancement rapide (PowerShell — Windows)}
\begin{lstlisting}[language=bash, backgroundcolor=\color{codebg}, basicstyle=\ttfamily\small\color{white}]
# Deploiement complet (14 conteneurs)
powershell -ExecutionPolicy Bypass -File .\run.ps1 deploy

# Demo complete pour le jury (7 etapes automatisees)
powershell -ExecutionPolicy Bypass -File .\run.ps1 full-demo

# Scaling horizontal (3 instances)
powershell -ExecutionPolicy Bypass -File .\run.ps1 scale 3

# Analytics et metriques
powershell -ExecutionPolicy Bypass -File .\run.ps1 analytics
\end{lstlisting}
\end{techbox}

\subsection{URLs d'Accès}

\begin{table}[H]
\centering
\begin{tabularx}{\textwidth}{l l X}
\toprule
\textbf{Service} & \textbf{URL} & \textbf{Description} \\
\midrule
Web UI (Dashboard)    & \url{http://localhost:8080}  & Interface principale (login requis) \\
API Gateway (Traefik) & \url{http://localhost:80}    & Point d'entrée API REST \\
Traefik Dashboard     & \url{http://localhost:8888}  & Visualisation des routes \\
Grafana               & \url{http://localhost:3000}  & Dashboards (admin/admin) \\
Prometheus            & \url{http://localhost:9091}  & Interface requêtes PromQL \\
Jaeger UI             & \url{http://localhost:16686} & Tracing distribué \\
MailHog               & \url{http://localhost:8025}  & Emails capturés \\
\bottomrule
\end{tabularx}
\caption{Points d'accès de la plateforme}
\end{table}

% =============================================================================
% SECTION 9 : KPI ET MÉTRIQUES
% =============================================================================
\section{KPI \& Métriques Clés}

\begin{figure}[H]
\centering
\begin{tikzpicture}[
  kpi/.style={rectangle, rounded corners=8pt, draw=#1!70, fill=#1!10,
              minimum width=3.5cm, minimum height=2.5cm, font=\small,
              align=center, drop shadow={shadow xshift=1pt, shadow yshift=-1pt}}
]
  \node[kpi=primary] (kpi1) {
    \textbf{\large MTTA}\\[5pt]
    Mean Time\\To Acknowledge\\[3pt]
    \faIcon{clock}~Temps moyen\\de prise en charge
  };
  \node[kpi=danger, right=0.8cm of kpi1] (kpi2) {
    \textbf{\large MTTR}\\[5pt]
    Mean Time\\To Resolve\\[3pt]
    \faIcon{tools}~Temps moyen\\de résolution
  };
  \node[kpi=secondary, right=0.8cm of kpi2] (kpi3) {
    \textbf{\large Corrélation}\\[5pt]
    Ratio alertes\\→ incidents\\[3pt]
    \faIcon{compress-arrows-alt}~Réduction\\du bruit
  };
  \node[kpi=accent, right=0.8cm of kpi3] (kpi4) {
    \textbf{\large Escalade}\\[5pt]
    \% incidents\\escaladés\\[3pt]
    \faIcon{level-up-alt}~Couverture\\astreinte
  };
\end{tikzpicture}
\caption{Les 4 KPI opérationnels suivis par la plateforme}
\end{figure}

% =============================================================================
% SECTION 10 : CONCLUSION
% =============================================================================
\section{Conclusion}

\begin{techbox}{Récapitulatif}
\textbf{OnCallNova} démontre qu'il est possible de construire une plateforme de gestion
d'incidents de niveau production avec :
\begin{itemize}[nosep]
  \item \textbf{5 microservices} communicant en gRPC + REST
  \item \textbf{14 conteneurs} Docker Compose orchestrés par Traefik
  \item \textbf{Observabilité complète} : Prometheus + Grafana + Jaeger + Loki
  \item \textbf{7 fonctionnalités bonus} implémentées (6 pleinement fonctionnelles)
  \item \textbf{Sécurité} : JWT, rate limiting, secrets externalisés
  \item \textbf{100\% local} — aucune dépendance cloud, déployable en une commande
\end{itemize}
\end{techbox}

\vspace{1cm}
\begin{center}
\begin{tikzpicture}
  \node[rectangle, rounded corners=10pt, draw=primary!60, fill=primary!5,
        minimum width=12cm, minimum height=2cm, font=\large, align=center] {
    \faIcon{rocket}~\textbf{Déploiement en une commande :}\\[5pt]
    \texttt{powershell -ExecutionPolicy Bypass -File .\textbackslash run.ps1 full-demo}
  };
\end{tikzpicture}
\end{center}

% =============================================================================
\end{document}
